\documentclass[journal,twocolumn,10pt, a4paper]{article}

\usepackage[top=0.95in, bottom=1in, left=0.95in, right=0.95in]{geometry}
\usepackage{array}
\usepackage[colorlinks,linkcolor={black},citecolor={blue!80!black},urlcolor={blue!80!black}]{hyperref}
\usepackage[parfill]{parskip}
\usepackage{lmodern}
\renewcommand*\familydefault{\sfdefault}
\setlength{\columnsep}{40pt}
\usepackage{watermark}

\usepackage{circuitikz}
\usetikzlibrary{circuits.logic.IEC,calc}
\usepackage{tikz}
\usepackage{amsmath}
\usepackage{setspace}
\usetikzlibrary{shapes, arrows, chains, decorations.markings,intersections,calc}
\usepackage{lipsum}
\usepackage{xcolor}
\usepackage{listings}
\usepackage{float}
\usepackage{titlesec}
\usepackage{amsmath}
\usepackage{tabularx}
\usepackage{algorithm2e}
\usepackage{./karnaugh-map}
\usepackage[utf8]{inputenc}
\usepackage{pgfplots}
\usepackage{listings}
\usepackage{placeins}
\documentclass[12pt]{article}
\usepackage{graphicx}
\usepackage{gensymb}
\usepackage[none]{hyphenat}
\usepackage{graphicx}
\usepackage{listings}
\usepackage[english]{babel}
\usepackage{graphicx}
\usepackage{caption}
\usepackage{hyperref}
\usepackage{booktabs}
\usepackage{array}
\usepackage{amsmath}
\usepackage{listings}
\usepackage{multirow}
\usepackage{blindtext}
\usepackage{capt-of}
\usepackage{circuitikz}
\usepackage{./karnaugh-map}
\usetikzlibrary{shapes.geometric}

\begin{document}

\lstset{ 
  language=C++,
  basicstyle=\ttfamily\footnotesize,
  breaklines=true,
  frame=lines}


\title{FPGA ASSIGNMENT}
\author{Tanyala Srihitha\\srihithatanyala@gmail.com}
\maketitle
\tableofcontents

\section{Problem}
(GATE EC-2022)\\

Q.19. Consider the 2-bit multiplexer(MUX) shown in the figure.For output to be the XOR of R and S,the values for $ W,X,Y$ and $Z$ are ?\newline
\begin{figure}[h]
\input{figs/fig1.tex}
\caption{mux}
\label{fig:1}
\end{figure}
\begin{enumerate}
\item $W = 0, X = 0, Y = 1, Z = 1$
\item $W = 1, X = 0, Y = 1, Z = 0$
\item $W = 0, X = 1, Y = 1, Z = 0$
\item $W = 1, X = 1, Y = 0, Z = 0$
\end{enumerate}

\section{Introduction}
	The above diagram is a 4:1 multiplexer where $W, X, Y, Z$ are the inputs of the multiplexer and $A$ is the output of the multiplexer.$R , S$ are the select lines of the multiplexer,which means:\newline
\begin{enumerate}
\item For $R = 0,S = 0$,the first input line $W$ is selected.
\item For $R = 0,S = 1$,the second input line $X$ is selected.
\item For $R = 1,S = 0$,the third input line $Y$ is selected.
\item For $R = 1,S = 1$,the fourth input line $Z$ is selected.
\end{enumerate}
Therefore,the resultant output expression of the multiplexer is $R'S'W + R'SX + RS'Y + RSZ$.

\section{Components}
\begin{table}[h]
\begin{tabular}{|p{2.25cm}|p{3cm}|p{2cm}|}
	\hline
	Component& Value& Quantity\\
	\hline
	Resistor &220 Ohm& 1\\
	\hline
	Arduino& UNO& 1\\
	\hline
	Seven Segment Display&  & 1\\              
	\hline                                     
	Decoder& 7447& 1\\                         
	\hline                                    
	Jumper Wires& M-M& 20\\                  
	\hline                                    
	Breadboard&  & 1\\                         
	\hline                                     
\end{tabular}


\caption{contents}
\label{table 1}
\end{table}


\section{Truth Table}
\begin{table}[h]
\begin{center}
\begin{tabular}{|p{1cm}|p{0.5cm}|p{0.5cm}|p{0.5cm}|p{0.5cm}|p{0.5cm}|p{0.5cm}|p{0.5cm}|}                            
\hline                                         
7447& $\overline{a}$ & $\overline{b}$ & $\overline{c}$ & $\overline{d}$ & $\overline{e}$ & $\overline{f}$ & $\overline{g}$\\          
\hline                                          
Display& a& b& c& d& e& f& g\\         
\hline                                        
\end{tabular}

\end{center}
\caption{truth table}
\label{table 2}
\end{table}
\pagebreak

\section{K-map}
The K-map for this truth table will be a two variable K-map and it will be as follows:
\begin{figure}[h]
\input{figs/fig2.tex}
\caption{k-map}
\label{fig2}
\end{figure}

\section{Hardware}                                      
\begin{enumerate}                                        
\item Set the GPIO pins: 2,3,4,5,6,7 of Vaman as inputs.
\item Set the GPIO pin 10,11 of Vaman as output.         
\item Read the input pins after connecting the Vcc and GND pins.                                                  
\item Verify the outputs using the truth table.         
\end{enumerate}

\section{Software}
The embedded code for the given circuit is \\
\lstinputlisting{codes/helloworldfpga.v}
\end{document}
